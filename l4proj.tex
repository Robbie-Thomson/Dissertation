% REMEMBER: You must not plagiarise anything in your report. Be extremely careful.

\documentclass{l4proj}

    
%
% put any additional packages here
%

\begin{document}

%==============================================================================
%% METADATA
\title{Social Acceptability of Novel Interaction Techniques}
\author{Robert B. Thomson}
\date{November 03, 2020}

\maketitle

%==============================================================================
%% ABSTRACT
\begin{abstract}
    Every abstract follows a similar pattern. Motivate; set aims; describe work; explain results.
    \vskip 0.5em
    ``XYZ is bad. This project investigated ABC to determine if it was better. 
    ABC used XXX and YYY to implement ZZZ. This is particularly interesting as XXX and YYY have
    never been used together. It was found that  
    ABC was 20\% better than XYZ, though it caused rabies in half of subjects.''
\end{abstract}

%==============================================================================

% EDUCATION REUSE CONSENT FORM
% If you consent to your project being shown to future students for educational purposes
% then insert your name and the date below to  sign the education use form that appears in the front of the document. 
% You must explicitly give consent if you wish to do so.
% If you sign, your project may be included in the Hall of Fame if it scores particularly highly.
%
% Please note that you are under no obligation to sign 
% this declaration, but doing so would help future students.
%
%\def\consentname {My Name} % your full name
%\def\consentdate {20 March 2018} % the date you agree
%
\educationalconsent


%==============================================================================
\tableofcontents

%==============================================================================
%% Notes on formatting
%==============================================================================
% The first page, abstract and table of contents are numbered using Roman numerals and are not
% included in the page count. 
%
% From now on pages are numbered
% using Arabic numerals. Therefore, immediately after the first call to \chapter we need the call
% \pagenumbering{arabic} and this should be called once only in the document. 
%
% Do not alter the bibliography style.
%
% The first Chapter should then be on page 1. You are allowed 40 pages for a 40 credit project and 30 pages for a 
% 20 credit report. This includes everything numbered in Arabic numerals (excluding front matter) up
% to but excluding the appendices and bibliography.
%
% You must not alter text size (it is currently 10pt) or alter margins or spacing.
%
% 
%==================================================================================================================================
%
% IMPORTANT
% The chapter headings here are **suggestions**. You don't have to follow this model if
% it doesn't fit your project. Every project should have an introduction and conclusion,
% however. 
%
%==================================================================================================================================
\chapter{Introduction}

% reset page numbering. Don't remove this!
\pagenumbering{arabic}


\section{Overview}
Social acceptability plays an important role in a person's willingness to use and interact with technology [Rico2010, Ahlstrom2014]. Novel interaction modalities and new interactive devices may initially be perceived as socially unacceptable due to being unfamiliar and maybe even unusual to onlookers---especially if it is not clear that a person is using a computing device. The aim of this dissertation is to evaluate the social acceptability of novel interaction techniques for mobile devices, focusing on the use of gestures for input. Device motion gestures (e.g., shaking a phone) and contactless gestures (e.g., waving above the screen) are widely supported by modern smartphones, but are yet to be widely adopted by users, especially contactless gestures.

This work also investigates a new aspect of perceived social acceptability, looking at the relationship between social acceptability and perceptions of how \textit{useful} an interaction may be to a user. Perceived usefulness or benefit to a user could influence perceptions of social acceptability by attributing a clear reason for \textit{why} a person is interacting with a device in a certain way. How this usefulness changes with exposure and familiarity will also be explored (e.g., to see if interactions become more acceptable as users begin to recognise their benefits).

The research begins with a preliminary survey to investigate current opinions on the use of novel interaction techniques and devices. Following this, an Android media player, with incorporated gesture input, was created. User evaluations over the course of one week were used to analyse the effect of using the novel interaction for the purpose of musical enjoyment over a period of time. Observations are made..... And conclusions are drawn.....


\section{Motivation}
Technological advances mean people are interacting in more ways with an increasing number of devices on a daily basis. In turn, people become more and more reliant on new devices and interaction modalities, because they offer more convenient ways of accessing services and information. They can be very efficient at helping us complete tasks in our everyday lives. 

People can often feel that using novel interaction techniques is socially unacceptable (i.e., that their use would be perceived negatively by other people). This is a serious issue for interaction designers and device manufacturers, as negative perceptions may slow down or prevent the adoption of new technologies. If an interaction is not deemed socially acceptable or does not adhere to social norms, the product seems likely to fail. Very few people will be happy to use it in their day to day lives if it might attract unwanted attention or negative perceptions from others.  It is often just taken for granted that if interaction is too different from the current technologies, or has an unclear use case, then the population will not get behind it. For example, Google Glass was negatively affected by lack of acceptance due to perceived privacy concerns, because interactions with the system and its intended applications were unclear.

Social acceptability is largely affected by the location of use and the `audience' of an interaction [Rico]. This is likely the case for certain circumstances of interactions. However, I believe that a persons familiarity with an interaction method will have just as large an impact, if not more, than these factors on their perceptions of acceptability. I believe the perceived use of an interaction technique will grow over time. As a result, the user may be more likely to perceive it to be socially acceptable. Possible factors of how quickly and likely these changes to perception are to occur are (1)~how useful a technique is and (2)~how little effort is required to use it, in comparison to the standard method of completing the desired task. These factors will be explored through this work.

\section{Aims}

This project aims to investigate the relationship between perceived usefulness and social acceptability of novel interaction techniques. It will do this through multiple approaches. First, a survey will investigate views on social norms and stigmas around effort and usefulness. It will investigate how perceived usefulness affects how people currently use or view novel interaction techniques. In particular, it will ask about gesture input, smart glasses and voice assistant interactions, as these technologies have varying levels of social acceptance currently.

After this, a media player will be implemented with integrated motion gesture input for interaction. Evaluation participants will be asked to use the mobile app over the course of a week and will be interviewed at various stages. This longitudinal study aims to see if users become familiar with the method of interaction over the evaluation time period. It is hypothesised that participants will find the gesture interactions more useful and easier to use over time, and in turn believe it will become socially acceptable to use in more situations. It is believed that this work will provide researchers with a new outlook on social acceptability: that the context of a situation is the boundary that must be met socially, but that acceptability is also affected by the user's perceived usefulness and familiarity with an interaction method.

If this belief is confirmed, it is hoped that interaction designers will be able to use the findings to create interaction techniques that scaffold a user's perceptions early on, e.g., by being easy to familiarise ones self with and recognise the utility of, as opposed to being completely socially acceptable ``out of the box''. 


%==================================================================================================================================
\chapter{Related Work}
\section{Overview}

This overview of related research begins with a brief look at novel interaction techniques and their social acceptability. It then explores social norms around usefulness and effort, to understand how these phenomena affect perceptions and behaviour.

\section{Novel Interactions}

In simple terms, an interaction with a device can be described as and action in which a users communicates input to the device, when the device provides output to the user or a combination of both through interfaces. In the field of Human Computer Interaction it is taken to be more of the way one experiences using the device and their perception of it[...Designing Interactions]. These interactions can use various techniques and are said to be novel when they are not commonly used or new to the users of the devices. These interactions can be through various parts of the device or completely encompass the device itself. 

Various pieces of literature have set guidelines for how interfaces and interactions should be designed [Guidelines...&designing interactions]. However, guidelines and practices change very regularly. There has always been an attention on invisible interfaces for personal use[...really really]. These prospects will be built on to understand how the effort required to interact dictates the invisibility and if this invisibility relates to how users feel others will perceive them in social situations.

\section{Social Acceptability}

Individuals make decisions on the social acceptability of their actions on a daily basis [ref?]. They do this by using their existing knowledge to assess their surroundings and consider if an action would be deemed acceptable [presentation of self...]. This is also the case when interacting with technology. Users evaluate their motivations and desire to use a particular technology and weigh this up against social factors and norms, to decide if an action is socially acceptable in the current usage context [Rico]; if not (e.g., due to concern it would look strange or attract unwanted attention) then users will not perform that interaction, or will fall back to an alternative modality. This is important for interaction and product designers because if a product or interaction technique is perceived to be socially unacceptable, users will not be willing to use them and adoption will be limited.

There are some widely given structures of how social acceptability of a device can be assessed. These will be considered in this work, although a slightly different approach will be taken as this project aims to investigate other factors that might affect acceptability. Rico et al.[....]. suggested that users see the interaction as a performance that must be done appropriately in various contexts. For example, in certain contexts (e.g., around others), users are likely to want to perform that interaction in a manner that does not attract attention; whereas in others, there is more of a focus on doing the performance correctly (e.g., when alone). Ahlstrom et al.[...]. explored the possibility of introversion or extroversion traits playing a part in what a user deems acceptable. For example the users personality can be what deems a task and its motivation acceptable in a given set of surroundings. A simpler approach is discussed by Sakamoto et al.[...], who believed that a user's aim when interacting is simply not to be noticed by others.

In contrast to these works, this dissertation will take the stance that users will feel more comfortable using an interaction technique if they believe (1)~it is a useful way to accomplish a task and (2)~they will be perceived as not putting in too much effort with respect to that outcome. In other words, an interaction's performance will be recognisable to others as an appropriate and convenient way of achieving an outcome, even if slightly unconventional (e.g., using contactless gestures to accomplish a task that could alternatively be accomplished using the touchscreen). Consequently, as a user gains experience with successful use of an interaction method, the more socially acceptable they will perceive it to be because they recognise it as an appropriate way of completing a task.

\subsection{Voice Assistants}

% EF: there are a lot of claims in here, although you'll need to provide references for them

Voice assistants, like the Amazon Echo and Google Nest, are devices that primarily make use of Voice User Interfaces for interaction. VUIs are an interaction modality that have divided many people. This is often due to their range of potential uses, reliability and acceptability. Issues with reliability often mean that, following attempts to control the device with their voice, they have to fall back to interacting with it through their mobile device. This means more effort is needed and interaction is less convenient than if the user had just used their phone in the first place. This causes frustration for the user and reduces their confidence in using it. People will perceive this as socially unacceptable because they don't want to be seen to be unable to complete the task confidently with minimal effort. This impact on acceptability has rarely been investigated, if a user if found to be more confident using an Interface method it could be that its easy to use and therefor will be perceived to be more socially acceptable.

\begin{figure}[!htb]
    \centering
    \includegraphics[width=0.75\textwidth]{images/VUI.jpg}
        \caption{Voice assistant using a VUI}
        \label{fig:syn1}
\end{figure}

A common method of gathering data about peoples views of a Voice Assistant is by analysing online product reviews. Experimental methods of this nature alone may not give a clear picture of the population as only certain groups of users may feel the need to leave a review, skewing the results. This research method was used to investigate users satisfaction and personification of the Amazon Echo [BFF...]. The authors of this work found some surprising results. They found that when technical errors were experienced, users were reluctant to continue using the device. It was assumed that these errors made people take notice of the device and its shortcomings, which caused people to not want to have to go to the bother of using it again. This dissertation will explore similar issues through its focus on perceived effort and the perception of `failing' to use a technology successfully. If technical issues require additional effort to overcome, then people may be put off from further use, because of frustration in overcoming the gulf of execution and the potential perception that they are `unable' to interact successfully.

A lab study undertaken by Myers et al.[Patterns...] investigated the causes and impacts of the frustration caused by VUIs. The experiment involved participants interacting with the device in a laboratory setting and required them to undertake set tasks on three occasions. Common errors made by participants and their levels of frustration were recorded. Users didn't take notice of times when the device accepted their attempts to interact, indicating it achieved invisibility and had the potential to be acceptable. However, when users had to put extra effort in, they became annoyed; when it reoccurred in later occasions, their frustration was reinforced. The acceptability of this reduced usefulness was not examined and could be explored in this research, as being frustrated is rarely seen as acceptable in other aspects of life. The Lab setting and set tasks limited these results as conclusions may not be strongly valid for how users would use the interaction techniques in the wild.

Myers et al.[Impact...] built on their previous work by exploring the limitations of the invisible nature of VUIs. Online responses and reviews were examined and an in-the-wild experiment was carried out to provide more ecologically valid results, in comparison to their previous lab study. This dual method will be attempted in my study of interaction methods, as both together can be ecologically valid by being in the wild while also gaining responses from large numbers of participants which may not have been possible to achieve due to the nature of using an interaction technique for an extended time.

From this initial investigation of voice assistants, it is clear that most studies focus on how people use them, with frustration and acceptability being emergent factors in the analysis. The reasons people may avoid using them have either not been explored, or been found by accident during analysis. This study will focus on the social reasons users may avoid using different novel interaction techniques through a combination of different experimental methods used by the above literature.

\subsection{Gestures}

Gestures are a non-verbal communication form, using meaningful body movements, posture, etc, to convey information. Gesture user interfaces similarly make use of body movements and postures for communication, in this case, to communicate intention to a computing device. Gestures can be sensed in several ways, impacting their form. The most common form is touch screen gestures, e.g., using multiple fingers, using varying pressures or tracing certain shapes. These are widely used by many devices and applications, and are generally socially acceptable. Device motion gestures (e.g., shaking or tilting a device for input) and mid-air gestures (e.g., waving a hand over a device or giving a thumbs-up) are two novel alternatives. These are less commonly used, although technology for sensing them is now commonplace in mobile devices. Motion and mid-air gestures are typically less socially acceptable than touchscreen interactions, likely in part due to being less common and requiring less subtle actions that might attract attention [Rico].

\begin{figure}[!htb]
    \centering
    \includegraphics[width=0.75\textwidth]{images/gestures.png}
        \caption{Potential actions for gesture input}
        \label{fig:syn1}
\end{figure}

The majority of previous literature has focused on gestures as not being socially acceptable in particular locations or in the presence of certain `audience' [Rico...]. The social acceptability of an interaction is generally assessed by asking a participant to carry out various tasks in different scenarios, or by asking them to imagine themselves performing an interaction in those scenarios. The participants views are either recorded throughout the tasks[are you comfortable ...] or collected at the end using an interview[...rico]. Users can also be examined on various metrics to understand how well they carry out the gesture[Rhythmic...]. Freeman et al.[...] detailed that mid-air gestures were very usable and completed easily when users were given instruction and found audio signals aligned well with their use due to the consistent eyes free nature of the features. This provides ground for the opportunity that users will be able to get more used to and more easily gain the benefits of Mid-Air Gestures. This will be explored with respect the the perceived social acceptance that is brought along with it.

Where gestures are performed is the focus of the study done by Rico et al. [...]. The author describes various types of gesture and explains why some of them are reasonable or should not be used. It is suggested that optimal gestures should mimic gestures that people may come across in their everyday life as they can be more familiar. Gestures that are found not to be effective are of an emblematic style as these may have preexisting meanings and could therefore be confused for their established connotations. Analysis will be done to maintain and solidify that using familiar gestures will make it easier for users to get used to using them, increasing their usefulness and therefore social acceptability.

Acceptance of an interaction method has also been found to have a strong correlation to where, with respect to their body, the user performs a gesture. Ahlström et al.[...] found that if carrying out the gesture took more than 6~seconds, the user tends to not see it as acceptable, because it is more likely to attract attention. Gestures that require body movement more than a foot away from the body are also considered less acceptable. Ahlström et al. [...] believe this to be purely for visual reasons as it may look unnatural to others and is more likely to attract unwanted attention. However, this could also indicate that users perceive an interaction technique to be socially unacceptable if they believe others around them will perceive them to be putting in an abnormal amount of effort to accomplish an interaction task. This will be explored through this dissertation research.

\subsection{Smart Glasses}

A wearable device is a broad term that covers devices such as Smart Glasses, Watches and Rings. Since these devices take the place of common fashion accessories they have an inherent need to be social acceptable. They must be  aesthetically pleasing, up to date with current trends and comfortable, all while facing the same interaction challenges of other devices, systems and applications. This study will focus only on the interactions with these types of devices, in particular Smart Glasses.

\begin{figure}[!htb]
    \centering
    \includegraphics[width=0.75\textwidth]{images/spectacles.jpg}
        \caption{Snapchat Spectacles}
        \label{fig:syn1}
\end{figure}

Research carried out by Chuah et al. [...] attempts to understand the factors that determine the widespread adoption of smartwatches by the population. It was concluded that visibility and perceived usefulness are large factors in this adoption. This will be explored to establish if it is due to the user wanting to appear visible and not using effort 
by others which makes them confident in using the devices - and similarly other novel interaction techniques - in social situations. Other studies focused on smart glasses are in appearance that a device and its interaction modalities being unobtrusive enhance its social acceptance. This will be elaborated to understand if this is common for other interaction techniques and to decide if, over time, users will better understand interaction techniques to become more comfortable using them, even if they are slightly obtrusive, due to a new found familiarity.

The Focus of this area of study will be directed towards the social acceptability and the use fullness of the Snapchat Spectacles as the product was available to me. The product and its interaction methods have been vastly speculated in the technical community [...Article?WhyDidTheyFail]. It is often deemed that the product was a failure and never took off. It seems that even when people bought the device, they rarely used it consistently and often stopped using it after as little as a week [...ArticleWhySnapchatSpectaclesFailed]. At this time there have been 3 iterations of the Spectacles and none of them have been close to being a common household item. The general consensus is that the fear brought about by other data glasses[...Don't Look at me].

Picture?


\section{Effort, Usefulness and Acceptability}

Perceived usefulness and effort required are is strongly linked. When solving an immediate problem, one will view it in comparison to their background of probable future problems[Human behaviour...].  If an individual sees an outcome as having less worth than the effort required to conduct the required task, it is likely they will not see it as useful and not want to do it . Kingsley also suggests that it is the socilal norm that a person will strive to expend the least amount of energy to solve a problem. In terms of using novel interaction techniques this can be translated to if the input to a device requires more energy than what the desired output is worth, they will not want to do it. Other literature [...perceived usefulness] strongly backs up this case in the context of personal user acceptance and has set guidelines to do so. Kingsley also believes that, "A person is socially treated according to the social signals he emits". This together with social norms of least effort, it will be explored if one has a concern that not only do they want to reduce effort to conserve energy, but also so it does not appear to others that they are wasting it. 

This claim is again backed up extracting knowledge from a study of attitudes among students[Student...]. If one wants to be accepted by society, they must give off the correct norm of socially signals accordingly. In this study it is found that from a young age, we learn that the social norm is to put in the least viable amount of effort. Therefore to be a part of the social norm we must adhere to this and not be seen to put in any effort or only as little as possible. Social signals coming from the effort expelled when providing input through novel interaction techniques will be explored to understand if this is common among these circumstances. It will be decided if invisibility of effort expelled is what is socially desirable, the same way it is when using a system to complete tasks for ones personal needs.

These implications will be analysed in the case of other novel interaction techniques and their social implications will be explored over time. It is thought that the more a user makes use of an interaction method, two main things may happen and will be observed. The users will become more used to using the device and feel as though the desired outcome requires less effort and become more useful. Additionally, they may come across more more useful functions that were not evident before use, increasing the output they will get by using it meaning it is acceptable to use more effort. These together will imply that a user will perceive the task to become more useful over time and and less effort in comparison will be expelled, making it more socially acceptable to do. Acceptance growing over time is examined in detail in The walkman effect[...]. This study details how over time as people use or do something it will gradually become more normal to do. The populations initial attitude towards the walkman was largely negative as it was a very out of the ordinary thing to use or be seen using. However, over time people became more attuned to it as its usefulness and ease of use that they became much more accepting to it as more and more others accepted it. Before long it became a norm. This effect will be investigated to understand if the same stands for novel interaction techniques.



%==================================================================================================================================
\chapter{Requirements Gathering}

\section{Outline}

In order to fully understand what needed to be carried out for this research, the task of gathering potential requirements was undertaken. Potential evaluation techniques and experiment aims were considered with respect to which novel interaction techniques could be analysed. Initially it was proposed that participants would be asked to use a type of novel interaction technique that they were not familiar with over a period of time to understand if their opinion on its social acceptability changed. Various novel interaction techniques were evaluated.

The potential for an evaluation of the participants usage and views of the Snapchat Spectacle wearable was explored, this would not involve a substantial amount of technical development so could be done easily in conjunction with another method. This evaluation could collect data from different participants detailing their frequency of use, opinion of usefulness and acceptability levels over some time. 

The opportunities of developing an application for either the Amazon Echo or Google Nest were investigated as both devices were available to me. This would explore the participants ways of using the device. For example, did they attempt to use speech all the time or did they simply control the speaker within the assistant and other devices using their mobile device as a remote. 

Another possible route that was pondered was to develop an app that would make use of some type of gestures. This could be Touch Gestures, Mid-Air Gestures or Device Motion Gestures. Participants could be evaluated on their usage and opinions of the features and interactions and their connecting ease of use, usefulness and social acceptability levels. It needed to be understood which of these pathways had the prospects of potential results. User preferences of gesture type was also an area of enquiry.

To gain an insight into the requirements, a User Survey was produced to understand what the populations current understanding on the subject area was. This provided a base understanding on what people perceive social acceptability to be and what effects it. Participants were also be questioned on their current experience with various novel interaction techniques. This needed to be known as if users currently have a vast experience using different types of novel interaction it would make any experiment results redundant. It was required that they have little experience at the start of the study in order for it to have the potential to grow with time and become more familiar with the novel interaction technique. An additional aim of the survey was to decipher if people had experience using gestures, and what their preferences would be for using them were, given the opportunity.

\section{Methodology}

A Survey was created and distributed to a sample of 24 participants, varying in age, gender and technical knowledge. Participants were briefed on the aims of the research as a whole, as well as the aims of the specific questionnaire. Consent was granted by all participants an they were informed that they may withdraw from the process at any time, they were directed to the relevant people had they had any questions throughout, as per the ethics checklist. 

At the start, participants where asked to detail their views on the how effort a task takes effects their likely-hood to do it. They were asked to elaborate on if they had ever avoided putting effort into a task for social reasons and if they were aware of a social slur surrounding this issue of avoiding applying excess effort to a task.

Following this, users were asked to gauge on a likert scale how socially acceptable using certain novel interactions in a specific location may be. This was then also asked for the case of the same task outcome and situation but simply using the touch screen alternative method on their mobile device, as opposed to the novel interaction technique. They were asked about the relative effort compared in the two circumstances. They were asked to detail if they frequently use any of the range of interaction techniques and given they opportunity in what circumstances they would happily use them and perceive it to be socially acceptable.

Finally, participants were asked to recommend what Gestures they would see fit for various functions that do not currently have associated with gestures in common devices. These were targeted to the functions that could be used in an app that could be created with integrated gesture input potential

\section{Results}
Participant responses to the survey questions were transferred to an Excel Spreadsheet, removing erroneous data in the process. Visualisations were developed and quantitative and qualitative analysis was done.

\subsection{Social norms related to perceived effort}
Half of the participants would not see it as appealing to undertake a task which required more effort than the output. An additional third thought that others, be it colleagues or fellow students or sports team members etc, would look at them badly for putting in this effort. This reinforces the idea that people believe there to be a social norm that you should avoid putting in extra effort when you don't need to. 

Only two of the participants said that they had never avoided appearing as though they had put in more effort into a task than what was required. Many of the participants gave examples of when they do this, some included; hiding how much you study for tests, how hard you they tried to get a job and even so not to appear like they need to put in extra work in case others think they are struggling. Building on the idea that people don't want to appear as though they are trying harder than necessary to complete a task. The results also introduce the idea that people do not want to look like they have to practice too much and put in too much effort to get good at something. People want to make it look like what they are doing has been done easily. 

The participants were presented with a name someone may call another if they feel as though they are going over and above what is expected them. Every single participant knew its connotations and were able to explain that they were negative. People are very aware of the norms of "trying too hard" to do something and the repercussions of doing so. Many people are sucked into this way of thinking and it has become accepted that one must not appear to be trying too hard to achieve something, be it their life goals or a small daily task.

\subsection{Perceptions of acceptability of novel interaction methods}.

There seemed to be some confusion understanding questions related to technologies participants were aware of and technologies participants use on a weekly basis. Some participants did not say that they were aware of some technologies, yet said that they used them on a weekly basis. Following the surveys closure to additional responses, participants were asked some additional questions about the survey, this included them detailing their understanding of this question. It was noted that some participants thought that Question 13 meant they were to select what technologies they were only aware of and had never used. It can therefore inherently be assumed that where a participant uses a technology on a weekly basis that they also are aware of it. With these adjustments, all participants stated they were aware of Voice assistants with 70.8\% using them on a weekly basis. Half of the participants were aware of Gestures with only a quarter claiming to use gestures on a weekly basis. All but 2 participants were aware of what Virtual reality was yet only 2 claiming to use it on a weekly basis. A total of 54.2\% of users were aware of the Snapchat Spectacles, with again only 2 claiming to use them on a weekly basis. No participant was unaware of any of these technologies but 6 participants did say that they did not use any of them weekly.

For questions relating to how users perceive using a shuffle feature on a mobile device, a total of 91.67\% of people rated using a mobile phones touch screen interface a five out of five for acceptability with the remaining , participants rating it a four. Suggesting people are very familiar with using a touch screen interface and see no reason to why it would be unacceptable to use. The average acceptability rating of using a device motion gesture was 4.25, slightly less than than the touch screen alternative rating of 4.92. This is not a substantial difference yet does still infer that there is less confidence around using gestures. When determining what users believe to use more effort there was a 45.8\% to 54.2\% split. This may infer that users have a very varying opinion of the use of touch screen versus motion gesture. This could be down to users unfamiliarity of the technology as 75\% of participants claimed that they do not use them on a weekly basis. It was stated that the most common reasons for a person not to use a gesture was if they were unhelpful (70.8\%) or not easy to use (66.7\%). Less than 17\% believed that company would effect this and only an eighth of users thought their location would change their mind not to use gestures. This was considerably less than the other interaction methods.

When asked about their perceptions of taking pictures and short videos when on a walk with friends, the average rating of how acceptable it is to take a picture with your phone was 4.58 on a five point scale, determining users find this very acceptable. On the other hand, participants gave wearing and using the wearable, Snapchat Spectacles, an average rating of less than 2. Inferring many people believe this to be a particularly unacceptable. Over 70\% of participants were under the impression that wearing the Spectacles would be more effort that simply taking your phone out and taking a picture with that. This could very well be the reason for people finding this method unacceptable. Users reasons to decide not to use spectacles backs this up. A total of 75\% of participants felt that the technology not being useful and taking a lot of effort with respect to the desired outcome would change their mind about using the technology. Again falling in line with two thirds of participants believing it not being easy to use to be a strong reason to avoid using the technology. Less than 30\% said that using the spectacles in unfamiliar company would make them rethink its use and only a quarter feeling that they would be affected by the location. 

Questioning participants about setting a timer produced the following responses. Setting an alarm on a mobile phone using the devices touch screen received a high average acceptability rating of 4.79 of of 5 from the participants. Doing the same task using a voice assistant received a moderate average rating of 3.54 of of 5. There was a further agreement with users in that using the voice assistant takes more effort, with only just over a third believing it is easier to use the voice assistant. This correlates well with the individual ratings as they seem to be very segregated. A third of participants rated the voice assistant a 5 for acceptability, a single parson rated it 2 and the remainder rated it a 3 out of 5. In general, where participants rated Voice assistants 2 or 3 they said it would take more effort to use than the phone and conversely for participants giving it a rating of 5. This had vary little variation suggesting there is a very divided view of Voice assistants. It was thought that this might be due to users familiarity of using VAs by either learning that they were not as easy to use as they appeared and used a lot of effort or by learning that they took little effort to use. Yet it was found that there did not seem to be a great correlation between these two attributes where the users were familiar. However, there was only one person that was not familiar with the technology and thought it would take less effort. While the other 6 that were unfamiliar with VAs thought it would take more effort than the mobile phone's touch screen alternative. Participants perception of VAs use being avoided due to it not being useful or not easy to use - just over half for both). However the company the user is in did seem to affect the participant's opinions more than in other technologies.

\subsection{User Gesture preferences and suggestions}
Users were provided with mobile device functions that are ordinarily controlled though touch screen or other hardware interaction. They were asked to provide an example of a gesture alternative that they felt would naturally fit that function. Responses were categorised into 5 categories; No Answer, Touch Gesture, Mid-Air Gesture, Device Motion Gesture, and Other. It must also be noted, during error correction, if a gestures was out of the reasonable technical scope or did not include a gesture, the response was counted as no answer unless any additional opinion is stated.

Participants were given the hypothetical task of adjusting the volume. Instead of pressing the volume up button, 8 participants opted for a touch gesture, 6 for Mid-Air and 5 for device motion. 5 participants gave no response. Instead of pressing the volume down button, 7 participants opted for a touch gesture, 7 for Mid-Air and 5 for device motion. 5 participants gave no response and one provided an alternative option. To adjust the volume people tend to want to use the touch screen and Mid air gestures over device motion gesture. This is likely to be due to the more controllable nature of these interaction types over definitive intervals. This would suit the nature of volume adjustments.

When presented with the opportunity to skip to the next song playing on the device, 6 participants were unable to provide and answer. 8 provided a touch based solution, 5 provided a Mid-Air Gesture, 4 opted for a device motion gesture and a single participant responded with a solution that was considered Other. Most answers were reverting back to various touch options, suggesting that the participants were failing to think of more innovative ideas of what gestures could represent this task. The number of answers for Mid-Air and Motion Gestures were both low with Mid-Air Gestures having a slight edge in preference.

Users were tasked with stating an alternative method of pausing and playing music. 7 participants supplied a response under the No Answer category and 2 under Other. 8 recommended a Touch gesture with only three favouring both Mid-Air and Device Motion style gestures. The number of participants that did not give an answer of a gesture increased. This suggests that The participants were running out of creative ideas to answer with, Most answers reverting back to various touch options, again reiterating that the participants losing interest. The number of answers were again low for Mid-Air and Motion Gestures.

Overall there is a lack of results of device motion gestures. It is thought that the results are biased against this due to many people being unfamiliar with the use of gestures. In general device motion gestures are fairly infrequently used within current technology so many of the participants may not be familiar with its potential and therefor failed to give an answer including it. Answers that did include device motion gestures tended to be more in detail which also suggests that it may be participants that are more familiar with the capabilities that suggested them.

\section{Conclusions}

It is believed that participants are  aware of the social norms of least effort. They were well informed of the social stigmas and potential negative outcomes inferred. This backs up the reasoning for exploring how these social norms may come into play when considering the social acceptability of novel interaction techniques.

It was decided that the focus of this research will have a focus on Mid-Air and Device Motion Gestures. Many factors were taken into account when making this decision. Voice assistant interactions were found to already be commonly used among the participants with many already being familiar with its use and having their views on them already set. Experimentation with  Snapchat Spectacles would not be reliable. Only one device was available and to unsure a reliable sample size, users would not be able to use them for a sufficient time to become familiar with them to detect a substantial change in opinion and visa versa. There seemed to be no resolve for this trade off. This method would also lack external validity as claims could only be made about this specific device due to its individuality. Gestures of such type were found to be the most practical and safe option during the Covid-19 pandemic. Users will be able to download an Android app which will implement Mid-Air and Device Motion Gestures without the need for any physical meeting or exchange of a device. 

Users appeared to understand these types of interactions without having experience using them. This would would enabled the potential for participants to increase their current use if the interaction technique and therefore introduce the ease of use and usefulness that will be found. These attributes effects on social acceptability would be measured over time. This lack of prior knowledge of these gesture types was shown in participants responses in the survey, they were unable to consistently provide reasonable gestures that could be used. Inspiration for the gestures that will be implemented was taken from survey responses and guidelines from other appropriate literature[...Rico]. Gestures will be made to be familiar to users without being emblematically of hand signals that may already have prior meanings in society. A shake style, Device Motion Gesture will be developed to initiate playing a random song. This is inspired by the shaking motion one may make before rolling dice, the result of each both being random. It will be possible to pause or play a song by holding a hand up in front if the device, taking inspiration from both user responses and the stop sign one may use (e.g a police officer stopping a vehicle). These share the common trait on directing something stop.


%==================================================================================================================================

\chapter{Technical Development}

\section{Overview}
Brief summary of development

\section{Platform}
Talk about the platforms considered to produce something. Android studio, spotify sdk, Google Home smart home action? Why was what was chosen done so(reasoning).

\section{Process}
Overview of how the app was created

\subsection{Design}
Explain how and why wireframes were developed

\subsection{Implementation}
What was developed, issues and how they were overcome

\subsection{Gestures}
How the gestures were decided and why. How they were created, why were they created that way.

\subsection{Assets}
How music was ethically and legally found, possible repercussions avoided

\section{Testing}
Tests that were undergone and why

%==================================================================================================================================

\chapter{Experiment /User Evaluation}

\section{Outline}
Participants would experience using the a novel interaction technique, that they had little familiarity of using previously. In this case it was decided that the novel interaction technique in question would be Device Motion Gestures and Mid-Air Gestures. This technique would be made available to them on their personal mobile device. Ensuring the user is familiar with the environment in which the interaction technique is build upon is very important to ensure no other external factors on unfamiliarity infer secondary effect. Participants would be provided an information leaflet and a consent forms to complete in line with the ethics checklist. Participants would then asked to complete surveys at a specific time frame during the experiments timeline. Collecting data throughout the experiment on the users experiences is important to understand opinions as time passes. Results were processed and searched for erroneous input, evidence of this was discarded from results. Results would then be analysed using various techniques.

\section{Ethical considerations}
General, in line with checklists, relation to covid.

forms.

info sheets.

usernames.

dangers.

\section{Interactions}
explain what interactions will be examined and why. Compare the mid air gestures and motion gestures with relation to space, symbolism etc.

\section{Method}
Explain the timeline of the experiment and the purpose of each part, what would be examined and how.

%==================================================================================================================================

\chapter{Results} 
As the participants progressed through their time experiencing the app and making use of the Gestures made available to them, they were providing data on their opinions through surveys. There responses were transferred to an Excel spreadsheet to be analysed. Before any analysis was carried out, the data was searched for any potential erroneous responses. Where possible these were amended accordingly or discarded where this was not possible. Where questions asked users to respond on a likert scale, Strongly Agree was encoded to a rating of 5, Agree to 4, Neutral to 3, Disagree to 2 and Strongly Disagree to 1. This was done to better represent users responses numerically to be included in visualisations. Users provided an unidentifiable keyword throughout to track their changes over time while still ensuring anonymity throughout. 

\section{Usage changes}
All users use an android phone as their personal device so were used to the operating system the app was build for. None of the participants believed their device regularly gave them the option to use mid air gestures. 50\% knew their device to give them options to provide options for motion gestures and only 25\% had used this interaction method and it was said to be less than weekly use. If provided the option to use mid air or motion gestures, a quarter said they would continue just to use the touch screen option with all remaining participants detailing that it would depend on if the interaction would require less effort than simply using the touch screen, one of which said it would also depend on social situation. All participants that said they would continue to use the touch screen were the same participants that knew their device had motion gestures but didn't use them. When asked what they would do if a function required the use of Motion or Mid-Air Gestures, with NO touch screen alternative, 25% said they would never use that function,% said they would use it no matter what, simply because they wanted to use the function, one said they would use it if it suited the situation, one said they would use it if it was useful, one said they would use it if it took little effort to use and one said they would use it if it fitted the social situation but only if it was useful.

With exception of 1-2 participants, users generally used the app 2-4 times after one day of use and again 2-4 further occasions after 3 days of use. Following this after one week, half the Participants used the app a further 5-10 times and the other half said they used it more than a further 10 times. Users tended to use the  Motion Gesture equal to or less than the amount of times they used the app, while using the mid air gesture equal to or more than the amount of times they used the app. One user asked after the survey for after 1 day of use if using the mid air gesture to pause and then play it again counted as one or two uses. After confirming that pausing an playing again only counted as one use of the feature due to its nature however if it was just use to pause or play and the button was used conversely it still counts as a use. This was then relayed to all other participants to confirm this was how they answered and would answer the usage questions in the surveys. Participants responded confirming that this is what they had assumed and answered all surveys accordingly.

\section{Participant Perceptions}

On average, after one day of use, participants rated the usefulness of  both the Device Motion Gesture and the Mid Air Gesture Functions 2.25/5. This increased to 3.125 and 3.975 respectively after 3 days of use and 4.25 and 4.375 respectively  after one week of use. This meant that the biggest increase for the Device Motion Gestures came between 3 days and one week of use and between 1 day and 3 days of use for the Mid Air Gestures. This closely correlates to when users were able to provide an occasion in which they thought using the novel interaction was more useful or less effort than simply pressing the tough screen. For the Device motion gesture, after 3 days only half the participants were able to provide an example, however all were able to provide one after 1 week.  For the Mid-air gesture, after 3 days all but one participant were able to provide an example, and again all were able to after a week.

The finding of new uses did not so closely correspond to the perceived ease of use in the same way that the usefulness factor did. The average rating of ease for the Device motion gesture had an between 1 day and 3 days of use, from 2.125 to 3.875 and then to 4.5 after a week of use. The average rating of ease for the Mid-air gesture raised from 2.375 to 3.75 between 1 and 3 days of use and then to 4.875 after a week of use. The increase in ease of use  in mid air gestures  increases more rapidly at the beginning of the experiment then at the end when compared to the mid air gestures.

Users appeared to find the gestures and their uses much less gimmicky over a very short time. All participants either disagreed that they were gimmicky or being neutral on the subject after only 3 days of use. 75\% of participants Strongly disagreed that the novel interaction techniques were gimmicky after a week of use. These changes strongly inversely correlate to the change in participants perceived social acceptability of both Device Motion and Mid air gestures. Participants responses showed, for both the Mid Air gestures and Device Motion Gestures, that they felt they would be uncomfortable using them in a situation that previous studies deemed unacceptable after only one day of use. This was expected. However, this perceived unacceptability began to change as the participants were able to use the novel interactions in their own time where they felt comfortable. No users disagreed  that they would confidently use either gesture technique in unfamiliar company or location after only 3 days of use. In the case of the Mid Air gesture, every participant strongly agreed that they would confidently use this novel interaction technique in unfamiliar company or an unfamiliar location. 


\section{Participant Opinions}
After one day of use, Five of the eight participants noted that they found it difficult to get used to using the gestures and those who did it were making the conscious decision to attempt to get in the way of using the gestures and get better at using them. Whereas, after a week of use, 5 of the eight participants strongly agreed that they had started us use the novel interaction techniques without consciously thinking about it. Strengthening this, all the participants strongly agreed that using the app over the week had made them more open to using these types of interaction  in more social situations than they would have before using the app and/or believe that mainstream applications don't currently and could make use of Device Motion and Mid-Air Gestures.

After a week of use, when asked to provide a circumstance where either gesture may not be acceptable, 75\% said that they may not use the device Motion Gesture in a cramped situation where it may result in physical contact with others or intruding in their personal spaces. One participant noted that they may not was to undertake exhaustive large movements when with their friends or in a meeting. One participant was unable to provide a circumstance.  Conversely to this, only one participant was able to provide an example of when it may be unacceptable to use the Mid-Ar Gesture. They stated that it may seem a bit odd in the beginning but it would become more normal is time passed. Users were asked if they could provide any other functions where these types of gesture interactions could be implemented. All participants responded with an example of an existing output with the gesture as an alternative way of controlling it, with half of the responses additionally providing reasoning that involved the gesture being used where the touch screen method of control used an excess of effort to complete. An example of this was to tilt to the side to remove all notifications as it can take time and effort to stretch to the top of the device to clear them all.

%==================================================================================================================================

\chapter{Evaluation/Analysis} 

%Implications should be made throughout 

\section{Usage change}

\section{User Perceptions}

%==================================================================================================================================
\chapter{Conclusion}  
\section{Summary}
Summarise the whole project for a lazy reader who didn't read the rest (e.g. a prize-awarding committee).

\section{Limitations}
Throughout this research, some boundaries and limits were encountered. The primary limitation that effected the research was implications of the Covid-19 Pandemic. 

\section{Implications}

\section{Future Work}


%==================================================================================================================================
\chapter{General Guidelines}

These points apply to the whole dissertation.

\subsection{Figures}
\emph{Always} refer to figures included, like Figure \ref{fig:relu}, in the body of the text. Include full, explanatory captions and make sure the figures look good on the page.
You may include multiple figures in one float, as in Figure \ref{fig:synthetic}, using \texttt{subcaption}, which is enabled in the template.



% Figures are important. Use them well.
\begin{figure}
    \centering
    \includegraphics[width=0.5\linewidth]{images/relu.pdf}    

    \caption{In figure captions, explain what the reader is looking at: ``A schematic of the rectifying linear unit, where $a$ is the output amplitude,
    $d$ is a configurable dead-zone, and $Z_j$ is the input signal'', as well as why the reader is looking at this: 
    ``It is notable that there is no activation \emph{at all} below 0, which explains our initial results.'' 
    \textbf{Use vector image formats (.pdf) where possible}. Size figures appropriately, and do not make them over-large or too small to read.
    }

    % use the notation fig:name to cross reference a figure
    \label{fig:relu} 
\end{figure}


\begin{figure}
    \centering
    \begin{subfigure}[b]{0.45\textwidth}
        \includegraphics[width=\textwidth]{images/synthetic.png}
        \caption{Synthetic image, black on white.}
        \label{fig:syn1}
    \end{subfigure}
    ~ %add desired spacing between images, e. g. ~, \quad, \qquad, \hfill etc. 
      %(or a blank line to force the subfigure onto a new line)
    \begin{subfigure}[b]{0.45\textwidth}
        \includegraphics[width=\textwidth]{images/synthetic_2.png}
        \caption{Synthetic image, white on black.}
        \label{fig:syn2}
    \end{subfigure}
    ~ %add desired spacing between images, e. g. ~, \quad, \qquad, \hfill etc. 
    %(or a blank line to force the subfigure onto a new line)    
    \caption{Synthetic test images for edge detection algorithms. \subref{fig:syn1} shows various gray levels that require an adaptive algorithm. \subref{fig:syn2}
    shows more challenging edge detection tests that have crossing lines. Fusing these into full segments typically requires algorithms like the Hough transform.
    This is an example of using subfigures, with \texttt{subref}s in the caption.
    }\label{fig:synthetic}
\end{figure}

\clearpage

\subsection{Equations}

Equations should be typeset correctly and precisely. Make sure you get parenthesis sizing correct, and punctuate equations correctly 
(the comma is important and goes \textit{inside} the equation block). Explain any symbols used clearly if not defined earlier. 

For example, we might define:
\begin{equation}
    \hat{f}(\xi) = \frac{1}{2}\left[ \int_{-\infty}^{\infty} f(x) e^{2\pi i x \xi} \right],
\end{equation}    
where $\hat{f}(\xi)$ is the Fourier transform of the time domain signal $f(x)$.

\subsection{Algorithms}
Algorithms can be set using \texttt{algorithm2e}, as in Algorithm \ref{alg:metropolis}.

% NOTE: line ends are denoted by \; in algorithm2e
\begin{algorithm}
    \DontPrintSemicolon
    \KwData{$f_X(x)$, a probability density function returing the density at $x$.\; $\sigma$ a standard deviation specifying the spread of the proposal distribution.\;
    $x_0$, an initial starting condition.}
    \KwResult{$s=[x_1, x_2, \dots, x_n]$, $n$ samples approximately drawn from a distribution with PDF $f_X(x)$.}
    \Begin{
        $s \longleftarrow []$\;
        $p \longleftarrow f_X(x)$\;
        $i \longleftarrow 0$\;
        \While{$i < n$}
        {
            $x^\prime \longleftarrow \mathcal{N}(x, \sigma^2)$\;
            $p^\prime \longleftarrow f_X(x^\prime)$\;
            $a \longleftarrow \frac{p^\prime}{p}$\;
            $r \longleftarrow U(0,1)$\;
            \If{$r<a$}
            {
                $x \longleftarrow x^\prime$\;
                $p \longleftarrow f_X(x)$\;
                $i \longleftarrow i+1$\;
                append $x$ to $s$\;
            }
        }
    }
    
\caption{The Metropolis-Hastings MCMC algorithm for drawing samples from arbitrary probability distributions, 
specialised for normal proposal distributions $q(x^\prime|x) = \mathcal{N}(x, \sigma^2)$. The symmetry of the normal distribution means the acceptance rule takes the simplified form.}\label{alg:metropolis}
\end{algorithm}

\subsection{Tables}

If you need to include tables, like Table \ref{tab:operators}, use a tool like https://www.tablesgenerator.com/ to generate the table as it is
extremely tedious otherwise. 

\begin{table}[]
    \caption{The standard table of operators in Python, along with their functional equivalents from the \texttt{operator} package. Note that table
    captions go above the table, not below. Do not add additional rules/lines to tables. }\label{tab:operators}
    %\tt 
    \rowcolors{2}{}{gray!3}
    \begin{tabular}{@{}lll@{}}
    %\toprule
    \textbf{Operation}    & \textbf{Syntax}                & \textbf{Function}                            \\ %\midrule % optional rule for header
    Addition              & \texttt{a + b}                          & \texttt{add(a, b)}                                    \\
    Concatenation         & \texttt{seq1 + seq2}                    & \texttt{concat(seq1, seq2)}                           \\
    Containment Test      & \texttt{obj in seq}                     & \texttt{contains(seq, obj)}                           \\
    Division              & \texttt{a / b}                          & \texttt{div(a, b) }  \\
    Division              & \texttt{a / b}                          & \texttt{truediv(a, b) } \\
    Division              & \texttt{a // b}                         & \texttt{floordiv(a, b)}                               \\
    Bitwise And           & \texttt{a \& b}                         & \texttt{and\_(a, b)}                                  \\
    Bitwise Exclusive Or  & \texttt{a \textasciicircum b}           & \texttt{xor(a, b)}                                    \\
    Bitwise Inversion     & \texttt{$\sim$a}                        & \texttt{invert(a)}                                    \\
    Bitwise Or            & \texttt{a | b}                          & \texttt{or\_(a, b)}                                   \\
    Exponentiation        & \texttt{a ** b}                         & \texttt{pow(a, b)}                                    \\
    Identity              & \texttt{a is b}                         & \texttt{is\_(a, b)}                                   \\
    Identity              & \texttt{a is not b}                     & \texttt{is\_not(a, b)}                                \\
    Indexed Assignment    & \texttt{obj{[}k{]} = v}                 & \texttt{setitem(obj, k, v)}                           \\
    Indexed Deletion      & \texttt{del obj{[}k{]}}                 & \texttt{delitem(obj, k)}                              \\
    Indexing              & \texttt{obj{[}k{]}}                     & \texttt{getitem(obj, k)}                              \\
    Left Shift            & \texttt{a \textless{}\textless b}       & \texttt{lshift(a, b)}                                 \\
    Modulo                & \texttt{a \% b}                         & \texttt{mod(a, b)}                                    \\
    Multiplication        & \texttt{a * b}                          & \texttt{mul(a, b)}                                    \\
    Negation (Arithmetic) & \texttt{- a}                            & \texttt{neg(a)}                                       \\
    Negation (Logical)    & \texttt{not a}                          & \texttt{not\_(a)}                                     \\
    Positive              & \texttt{+ a}                            & \texttt{pos(a)}                                       \\
    Right Shift           & \texttt{a \textgreater{}\textgreater b} & \texttt{rshift(a, b)}                                 \\
    Sequence Repetition   & \texttt{seq * i}                        & \texttt{repeat(seq, i)}                               \\
    Slice Assignment      & \texttt{seq{[}i:j{]} = values}          & \texttt{setitem(seq, slice(i, j), values)}            \\
    Slice Deletion        & \texttt{del seq{[}i:j{]}}               & \texttt{delitem(seq, slice(i, j))}                    \\
    Slicing               & \texttt{seq{[}i:j{]}}                   & \texttt{getitem(seq, slice(i, j))}                    \\
    String Formatting     & \texttt{s \% obj}                       & \texttt{mod(s, obj)}                                  \\
    Subtraction           & \texttt{a - b}                          & \texttt{sub(a, b)}                                    \\
    Truth Test            & \texttt{obj}                            & \texttt{truth(obj)}                                   \\
    Ordering              & \texttt{a \textless b}                  & \texttt{lt(a, b)}                                     \\
    Ordering              & \texttt{a \textless{}= b}               & \texttt{le(a, b)}                                     \\
    % \bottomrule
    \end{tabular}
    \end{table}
\subsection{Code}

Avoid putting large blocks of code in the report (more than a page in one block, for example). Use syntax highlighting if possible, as in Listing \ref{lst:callahan}.

\begin{lstlisting}[language=python, float, caption={The algorithm for packing the $3\times 3$ outer-totalistic binary CA successor rule into a 
    $16\times 16\times 16\times 16$ 4 bit lookup table, running an equivalent, notionally 16-state $2\times 2$ CA.}, label=lst:callahan]
    def create_callahan_table(rule="b3s23"):
        """Generate the lookup table for the cells."""        
        s_table = np.zeros((16, 16, 16, 16), dtype=np.uint8)
        birth, survive = parse_rule(rule)

        # generate all 16 bit strings
        for iv in range(65536):
            bv = [(iv >> z) & 1 for z in range(16)]
            a, b, c, d, e, f, g, h, i, j, k, l, m, n, o, p = bv

            # compute next state of the inner 2x2
            nw = apply_rule(f, a, b, c, e, g, i, j, k)
            ne = apply_rule(g, b, c, d, f, h, j, k, l)
            sw = apply_rule(j, e, f, g, i, k, m, n, o)
            se = apply_rule(k, f, g, h, j, l, n, o, p)

            # compute the index of this 4x4
            nw_code = a | (b << 1) | (e << 2) | (f << 3)
            ne_code = c | (d << 1) | (g << 2) | (h << 3)
            sw_code = i | (j << 1) | (m << 2) | (n << 3)
            se_code = k | (l << 1) | (o << 2) | (p << 3)

            # compute the state for the 2x2
            next_code = nw | (ne << 1) | (sw << 2) | (se << 3)

            # get the 4x4 index, and write into the table
            s_table[nw_code, ne_code, sw_code, se_code] = next_code

        return s_table

\end{lstlisting}

See the file \texttt{guide\_to\_visualising.pdf} for further information and guidance.

\begin{figure}
    \centering
    \includegraphics[width=1.0\linewidth]{images/boxplot_finger_distance.pdf}    

    \caption{Average number of fingers detected by the touch sensor at different heights above the surface, averaged over all gestures. Dashed lines indicate
    the true number of fingers present. The Box plots include bootstrapped uncertainty notches for the median. It is clear that the device is biased toward 
    undercounting fingers, particularly at higher $z$ distances.
    }

    % use the notation fig:name to cross reference a figure
    \label{fig:boxplot} 
\end{figure}


% 
%==================================================================================================================================
%  APPENDICES  
\begin{appendices}

\chapter{Appendices}

Typical inclusions in the appendices are:

\begin{itemize}
\item
  Copies of ethics approvals (required if obtained)
\item
  Copies of questionnaires etc. used to gather data from subjects.
\item
  Extensive tables or figures that are too bulky to fit in the main body of
  the report, particularly ones that are repetitive and summarised in the body.

\item Outline of the source code (e.g. directory structure), or other architecture documentation like class diagrams.

\item User manuals, and any guides to starting/running the software.

\end{itemize}

\textbf{Don't include your source code in the appendices}. It will be
submitted separately.

\end{appendices}

%==================================================================================================================================
%   BIBLIOGRAPHY   

% The bibliography style is abbrvnat
% The bibliography always appears last, after the appendices.

\bibliographystyle{abbrvnat}

\bibliography{l4proj}

\end{document}
